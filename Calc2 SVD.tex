\documentclass{article}
\usepackage[utf8]{inputenc}
\usepackage[english]{babel}
\usepackage{mathtools}
\usepackage[]{amsthm} %lets us use \begin{proof}
\usepackage[]{amssymb} %gives us the character \varnothing
\newcommand{\lm}{\lambda}
\newcommand{\sm}{\lambda}



\title{Calculus and Linear Algebra II \\ SVD}
\author{Getuar Rexhepi}
\date\today

\setlength\parindent{24pt}


\begin{document}
\maketitle %This command prints the title based on information entered above

%Section and subsection automatically number unless you put the asterisk next to them.

\section {Singular Value Decomposition}
Theorem: \\
Any mxn matrix A has a single value decomposition (SVD) $$A = U \Sigma V^+$$ where U is the unitary mxm and V unitary nxn.\\
$\Sigma- mxn$ matrix with only non negative values $\sigma_i$ on the diagonal and zeros everywhere else. The $G_i$ are called singular values of A, they are uniquely determined by A.\\


$$ A  = U \Sigma V^+ $$
$$ A^+A = ( U \Sigma V^+ )^+( U \Sigma V^+ ) = V \Sigma^+ U^+ U \Sigma V^+ =  V \Sigma^+\Sigma V^+ $$

$$ AA^+ =  (U \Sigma V^+) ( U \Sigma V^+ )^+ =  U \Sigma V^+ V \Sigma V^+ U^+ = U \Sigma^+ \Sigma^+ U^+ $$ 

$ \Sigma^+ \Sigma $ and $\Sigma \Sigma^+ $ are diagonalizable with $\sigma_1^2   ... \sigma_1^2  $ (k = min(m,n)\\

$$ \begin{bmatrix}
1 & & & &  &\\
&  1& & &  &\\
& & 1& &  &\\
& & & 1&  &\\
& & & &  1&\\
& & & &  &1 \\
\end{bmatrix}
$$

$ A^+ A$ and $AA^+$ are diagonalizable by a unitary matrix \\

We can conclude:\\

$\sigma_1^2 \dots \sigma_k^2 $ are eigenvalues of both $ A^+ A$ and $AA^+$\\

V  has orthonormal eigenvectors of $ A^+A$ as columns.\\

 U has orthonormal eigenvectors of $AA^+$ as columns.\\

$$ Ax = y $$ 

$$A=  \begin{bmatrix}
2 & 0\\
0 &3  \\
\end{bmatrix}
$$
But what if :
$$A=  \begin{bmatrix}
2 & 1\\
0 &2  \\
\end{bmatrix}
A^+ =  \begin{bmatrix}
2 & 0\\
1 &2  \\
\end{bmatrix}$$
$$
A^+A = \begin{bmatrix}
2 & 0\\
1 &2  \\
\end{bmatrix}
\begin{bmatrix}
2 & 1\\
0 &2 \\
\end{bmatrix}
=\begin{bmatrix}
4 & 2\\
2&5  \\
\end{bmatrix}
$$

$$ det(\begin{bmatrix}
4-\lambda& 2\\
2 &5- \lambda  \\
\end{bmatrix})
=$$

$$ (4-\lambda)(5-\lambda)-4  =20 + \lambda^2-9\lambda-4 = \lambda^2 - 9\lambda+16=0$$
From this we have the singular values:
$$ \sigma_1, \sigma_2 = \sqrt{\frac{9\pm\sqrt{17}}{2}}$$

\section {Linear Differential Equations with constant coefficients}
$$ a_n y^n + a_{n-1}y^{n-1} + ... + a_1 y' + a_0 y = f(t) $$

Homogeneous Equations :\\

$$a_n y^n + a_{n-1}y^{n-1} + ... + a_1 y' + a_0 y = 0$$

The General solution is of form :\\
$$ e^{\lm t} \implies a_n \lm^n  e^{\lm t} +  a_{n-1}\lm^{n-1}e^{\lm t}+... + a_1 \lm e^{\lm t} + a_0 e^{\lm t} $$


$$ (a_n \lm^n +  a_{n-1}\lm^{n-1}+... + a_1 \lm + a_0) e^{\lm t} = 0$$

Example1 :\\

$$ y''' - y' = 0 $$

$$ \lm^3 - \lm = 0 \implies \lm = 0,+1,-1$$

$$ y(t) = C_1e^{0t}+C_2e^{t}+C_3e^{-t} = C_1+C_2e^{t}+C_3e^{-t}$$

Harmonic Oscillator:\\
$$ y''+y = 0$$
$$ \lm^2 + 1 \implies \lm = \pm i$$

First case:
$$\lm - real, distinct \implies C_1e^{\lm_1t} +...+C_ne^{\lm_n t} $$

Second case:
$$\lm - a \pm ib \implies y(t) = C_1 e^{at}\cos(bt)+c_2e^{at}\sin(bt)$$

Third case:
$$y''' + 3y'' +3y' + y = 0$$
$$ \lm^3+3\lm^2+3\lm + 1 = 0 \implies (\lm+1)^3 = 0$$
$\lm=-1$ with algebraic mult. = 3
$$ y(t) = c_1e^{-t}$$

$$ y(t) = ( C_1+ C_2t+ C_3t^2) e^{-t}$$

Fourth case:
$$\lm = a\pm ib $$
with multiplity m
$$ y(t) = ( C_1+ C_2t+ ... C_mt^{m-1})e^{at}\cos(bt) + (D_1+D_2t...+D_mt^{m-1})e^{at}\sin{bt}$$


Dealing with the right hand side:\\

$$a_n y^n + a_{n-1}y^{n-1} + ... + a_1 y' + a_0 y = f(t)$$

f(t) : polynomials, exponentials, sin and cos.\\ 

\newpage

If we have:\\
$$ y''' - y' = t^2e^{2t}+\sin(t) $$
$$y_{gen}(t) =y_{gen homo}(t) + y_{gen inhomo}(t) $$

From above $\lm =2 $ and for the sin we have: $\lm = \pm i$

Now the solution will be:
$$ e^{2t}(A_1t^2A_2t+A_3) + A_4\cos(t)+A_5\sin(t) $$

$$ y'= -A_4\sin(t)+A_5\cos(t)$$
$$ y''= -A_4\cos(t)+A_5\sin(t)$$
$$ y'''= A_4\sin(t)+A_5\cos(t)$$

$$ A_4\sin(t)-A_5\cos(t)+A_4\sin(t)+A_5\cos(t) = \sin(t)$$

$$ -A_5 = 0 $$
$$ 2A_4 = 1 $$ 
$$A_4 = \frac{1}{2}, A_5 = 0$$








\end{document}